%% origial https://giuliavattuone.com/papers/jmp/

\documentclass[aspectratio=169]{beamer}
\setbeamersize{text margin left=0.5cm, text margin right=1cm}
%\setlength{\textheight}{23cm}
%\setlength{\textwidth}{15cm} 
\usepackage{babel}
\usepackage{xcolor}
\usepackage{amssymb}
\usepackage{tikz}
\usepackage{hyperref}

% navigation
\setbeamertemplate{navigation symbols}{}
\setbeamertemplate{headline}{}
\setbeamertemplate{footline}
{
	\begin{tikzpicture}[remember picture,overlay]
		\ifnum\insertframenumber>1
		\node[anchor=south east] at (current page.south east) {\scriptsize \insertframenumber/\inserttotalframenumber};
		\fi
	\end{tikzpicture}
}

% colors
\definecolor{indigo}{RGB}{13, 61, 86}
\definecolor{dgray}{RGB}{85,95,97}
\definecolor{borderColor}{RGB}{13, 61, 86}
\definecolor{bgColor}{RGB}{194, 207, 213}
\setbeamercolor{item}{fg=indigo}
\setbeamercolor{frametitle}{fg=indigo}
\setbeamercolor{section in toc}{fg=indigo}
\setbeamercolor{section in head/foot}{fg=indigo}
\setbeamercolor{title}{fg=indigo}
\setbeamercolor{date}{fg=dgray}

% button
\setbeamertemplate{button}{\tikz
		\node[
		inner xsep=8pt,
		inner ysep=2pt,
		draw=indigo,
		fill=bgColor,
		rounded corners=5pt
		]  {\small \insertbuttontext};}
\setbeamercolor{button}{bg=bgColor,fg=black}
\setbeamercolor{button border}{fg=indigo}

% item 
\setbeamertemplate{itemize item}[diamond] 
\setbeamercolor{itemize item}{fg=indigo} 
\setbeamerfont{itemize item}{size=\normalsize} 
\setlength{\leftmargini}{0.3em}
\setlength{\leftmarginii}{0.6em} 
\setlength{\leftmarginiii}{0.9em}  

% font 
\useoutertheme{default}
\usefonttheme{serif}
\setbeamerfont{title}{family=\rmfamily, shape=\scshape}
\setbeamerfont{frametitle}{family=\rmfamily, shape=\scshape}
\setbeamerfont{section in toc}{family=\rmfamily, shape=\scshape}

% title page
\setbeamertemplate{title page}
{
	\begin{center}
		\vskip5em
		{\usebeamerfont{title}\color{indigo}\inserttitle\par}
		{\color{indigo}\rule{0.6\textwidth}{0.4pt}} % Horizontal rule
		\vskip1.5em
		{\usebeamerfont{author}\color{black}\insertauthor\par}
		\vskip1em
		{\usebeamerfont{institute}\color{black}\insertinstitute\par}
		\vskip1em
		{\usebeamerfont{date}\color{dgray}\insertdate\par}
	\end{center}
}

\title{Worker Sorting \\ and the Gender Wage Gap}
\author{Giulia Vattuone \\ SOFI, Stockholm University}
\institute{Society for Economic Dynamics Annual Meeting}
\date{Barcelona, June 27-29, 2024}

% define 
\newcommand{\customtitle}[1]{
	\begin{flushleft}
		\usebeamerfont{frametitle} 
		\usebeamercolor[fg]{frametitle}
   		\vspace{2pt}
		{\scalebox{0.8}{#1}}
		\vspace{2pt}
	\end{flushleft}
}


\AtBeginSection[]{
	\begin{frame}{Outline}
		\tableofcontents[currentsection,currentsubsection]
		%\addtocounter{framenumber}{-1}
	\end{frame}
} 

\begin{document}
\frame{\titlepage}
\begin{frame}
	\frametitle{Introduction}
	\begin{itemize}
		\item[$\diamond$] This result is confirmed across multiple countries.
		\item[$\diamond$] Not due to lack of skills or experience
	\end{itemize}
\end{frame}

\begin{frame}
	\frametitle{This paper}
	\customtitle{Research Question:} 
	
	\customtitle{Data and Method:}

\end{frame}

\begin{frame}{Contribution}
\begin{itemize}
	\item[$\bullet$] \textit{Literature that \textbf{quantifies the sorting component} of the gender wage gap}
	\begin{itemize}
		\item[$\diamond$] {\footnotesize \textcolor{dgray}{(Card et al., 2016; Cardoso et al., 2016; Casarico and Lattanzio, 2024; Palladino et al., 2021)}}
	\end{itemize}
$\Longrightarrow$	I gauge the relative importance of key mobility components driving it
\end{itemize}
\end{frame}

\begin{frame}
	\frametitle{Outline}
	\tableofcontents
\end{frame}

\section{1) Revealed preference approach and Statistical model}
\begin{frame}{Revealed preference approach}
	Data on observed firm-to-firm transitions are informative about:
\end{frame}

\section{2) Data}
\begin{frame}{French Matched Employer-Employee Data}
	Data source
\end{frame}

\section{3) Clustering results}
\begin{frame}[label=refthis1]{Latent types capture diverse career trajectories}
	Stagnant wages irrespective of experience levels for low-wage worker types
\end{frame}

\section{4) Worker Sorting and the Gender Wage Gap}
\begin{frame}{Worker Sorting}
	Sorting is the stationary allocation of worker types and firm classes
\end{frame}

\section{5) Conclusions}
\begin{frame}[label = eatimation]{Estimation}
\begin{itemize}
\item[$\diamond$] \textcolor{indigo}{\textbf{First step:}} link to website: \href{https://www.microsoft.com/zh-cn/}{\beamerbutton{K-Means Algorithm}}\\

$\rightarrow$ firms in same class are similar in gender-specific wage distributions, female share, size

\vspace{1cm}

\item[$\diamond$] \textcolor{indigo}{\textbf{Second step:}} conditional on frm classes, EM algorithm to estimate parameters and
classify workers\\

$\rightarrow$ Likelihood function is non-linear in the parameters

$\rightarrow$ link to another frame:\hyperlink{refthis1}{\beamergotobutton{{EM+MM Algorithm}}}\quad \textcolor{dgray}{(Lentz, Piyapromdee, and Robin, 2023)}
\end{itemize}
\vskip 3cm
\centering
\hyperlink{sort}{\beamergotobutton{Data used to discretise}}
\quad
\href{https://www.microsoft.com/zh-cn/}{\beamerbutton{Model fit}}
\end{frame}

\section{6) Worker Sorting and the Gender Wage Gap}
\begin{frame}[label = sort]{Worker Sorting}
	Sorting is the stationary allocation of worker types and firm classes \hyperlink{eatimation}{\beamerreturnbutton{Data used to discretise}}
	
	$$\mathcal{L}_i(\theta|l_i,x_{i1},C)=\Pr\Big(l_i,k_{j(i,1)}\mid x_{i1}\Big)$$
\end{frame}
\end{document}
